% LaTeX file for resume
% This file uses the resume document class (res.cls)
\documentclass{res}
\include{private}

\name{\Huge{Celia Siu}\\[4pt]}
\address{%
  \centerline{%
    \faEnvelopeO{ \email }%
    \faMobile{ \cellphone }}\\%
    \centerline{ \where $|$ \url{csiu.github.io} }\\
  \rule{\textwidth}{0.4pt}{}}


% Customize spacing
\newsectionwidth{0pt}  % So the text is not indented under section headings
\topmargin=-0.5in      % start text higher on the page

% Customize the header
\usepackage{fancyhdr}
\pagestyle{fancy}                  % set pagestyle for document
\renewcommand{\headrulewidth}{0pt} % suppress line drawn by default by fancyhdr
% - Adjust spacing
\setlength{\headheight}{24pt} % allow room for 2-line header
\setlength{\headsep}{24pt}    % space between header and text
\setlength{\footskip}{0pt}
% - Define headers and footers
\rhead{}% {\it csiu}\\{\it p. \thepage} }
\cfoot{%
  \rule{\textwidth}{0.4pt}{}
  \faLinkedinSquare{ \url{https://ca.linkedin.com/in/celiasiu} }
  \faGithubSquare{ \url{http://github.com/csiu} }\\
  \faStackOverflow{ \url{http://stackoverflow.com/users/2364901/csiu} }
}

% Enable use of packages
%\usepackage{helvetica} % uses helvetica postscript font (download helvetica.sty)
%\usepackage{newcent}   % uses new century schoolbook postscript font
\usepackage{fontawesome}

% Enable link
\usepackage{hyperref}

% Enable condition ifthenelse
\usepackage{ifthen}

% Enable spacing
\usepackage{scrextend}

% ------------------------------------------------------------------------------
% Customizations - pt.2

% - Update section header
\renewcommand{\section}[1]{%
  \vspace{0.3in}%
  \centerline{\uppercase{\bf{#1}}}%
  \vspace{-6pt}}

% - Award line
\newcommand{\lineaward}[2]{%
  #1 \faTrophy{} #2\\[1ex]}

% - Presentation lines
\newcommand{\linepresentation}[3]{%
  #3 \hfill #1\\[.75ex]}

% - Employment header lines
\newcommand{\linehead}[2]{%
  {\bf #1} \hfill #2\\}
\newcommand{\linetitle}[1]{%
  {\sl #1}}

% - Volunteer line
\newcommand{\linevolunteer}[3]{%
  {\sl #3}, {\bf #2} \hfill #1}%\\[.5ex]}

% - define new environment for publications
\newenvironment{publicationlists}{%
  \begin{itemize}\leftmargin=3em \itemindent=-1em \itemsep=2pt%
  }{%
  \end{itemize}}

% - Define achievement & awards section
\newcommand{\awarditem}[3]{%
  \ifthenelse{\equal{#2}{}}
    {{\bf #1}}
    {{\bf #1}\hfill#2}
    \\ \vspace{-1em}
    \begin{addmargin}[2em]{3em}% 2em left, 3em right
    #3
    \end{addmargin}
    \vspace{-1em}
    }

% - update itemize have smaller itemsep
\let\olditemize\itemize
\renewcommand{\itemize}{%
  \olditemize\addtolength{\itemsep}{-4pt}}

% - url to roman (serif) font
\urlstyle{rm}


% ==============================================================================
\begin{document}
\begin{resume}

% \thispagestyle{empty} % no header on first page

% ------------------------------------------------------------------------------
\vspace{-1em}
\begin{center}
  {\it As a MSc in Computational Biology, I do Data Science work applied to biological data. I am passionate about Data Science, Machine Learning, Text Mining, and Data Visualization. Fluent in Python and R.}
\end{center}
\vspace{-1.5em}

\section{Projects in Data science}

\linehead{100 Days of Code}{Present}
\linetitle{See my technical blog: \url{http://csiu.github.io/blog/tag/100daysofcode}}
\begin{itemize}
  \item Used machine learning methods from Scikit-learn to build models and enter Kaggle competitions
  \item Acquired and analyzed data using Jupyter Notebooks in Python (eg. Pandas, Matplotlib)
\end{itemize}

\linehead{MSc thesis: Characterization of the human thyroid epigenome}{Sept 2015 -- Present}
\linetitle{Michael Smith Genome Sciences Centre, Jones Lab}
\begin{itemize}
  \item Modelled chromatin states from epigenetic features using hidden Markov models
  \item Developed a new quantitative metric for model selection
\end{itemize}

\linehead{Literature-based knowledge discovery of biomedical text}{May 2015 -- Aug 2015}
\linetitle{Michael Smith Genome Sciences Centre, Jones Lab}
\begin{itemize}
  \item Developed text mining/machine learning approaches to prioritize therapeutics from unstructured text
  \item Created data visualizations in R and JavaScript (D3) from data stored in databases
\end{itemize}

\section{Work Experience}

{\bf Computational Biology Research Assistant} \hfill Sept 2013 -- Aug 2014\\
{\it BC Cancer Agency, Shah Lab}
\begin{itemize}
  \item Wrote Bash, Python and R scripts to preprocess data, develop pipelines, and wrap third party software
  \item Produced data visualizations and limited text suitable for scientific manuscripts
  \item Initiated pipelines on compute clusters to generate data for downstream analysis
  \item Following an internship, promoted to full-time
\end{itemize}

{\bf Undergraduate Research Student} \hfill May 2013 -- Aug 2013\\
{\it UBC Department of Statistics}
\begin{itemize}
  \item Worked with collaborators to develop an R package using Git and Github
  \item Wrote unit tests in accordance with best coding practices in software development
\end{itemize}

\section{Volunteer experience}
\linevolunteer{Oct 2016 \& Feb 2017}{%
  Software Carpentry workshop}{%
  Volunteer}
\begin{itemize}
  \item Troubleshot and helped students learn Git, GitHub, Bash, Python, and Jupyter Notebook
\end{itemize}

\section{Education}
\linehead{Master of Science, Computational Biology}{Sept 2014 -- Present}
University of British Columbia, Vancouver, BC
\begin{itemize}
  \item 10\% acceptance rate and awarded \$50,000+ in scholarships
  \item Took graduate-level courses in Machine Learning, Data Mining, and Design and Analysis of Algorithms
\end{itemize}
\pagebreak

\end{resume}
\end{document}
