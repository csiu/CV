% LaTeX file for resume
% This file uses the resume document class (res.cls)
\documentclass{res}
\include{private}

\name{\Huge{Celia Siu}\\[4pt]}
\address{%
  \centerline{%
    \faEnvelopeO{ \email }%
    \faMobile{ \cellphone }}\\%
    \centerline{ \where $|$ \url{csiu.github.io} }\\
  \rule{\textwidth}{0.4pt}{}}


% Customize spacing
\newsectionwidth{0pt}  % So the text is not indented under section headings
\topmargin=-0.5in      % start text higher on the page

% Customize the header
\usepackage{fancyhdr}
\pagestyle{fancy}                  % set pagestyle for document
\renewcommand{\headrulewidth}{0pt} % suppress line drawn by default by fancyhdr
% - Adjust spacing
\setlength{\headheight}{24pt} % allow room for 2-line header
\setlength{\headsep}{24pt}    % space between header and text
\setlength{\footskip}{0pt}
% - Define headers and footers
\rhead{}% {\it csiu}\\{\it p. \thepage} }
\cfoot{%
  \rule{\textwidth}{0.4pt}{}
  \faLinkedinSquare{ \url{https://ca.linkedin.com/in/celiasiu} }
  \faGithubSquare{ \url{http://github.com/csiu} }\\
  \faStackOverflow{ \url{http://stackoverflow.com/users/2364901/csiu} }
}

% Enable use of packages
%\usepackage{helvetica} % uses helvetica postscript font (download helvetica.sty)
%\usepackage{newcent}   % uses new century schoolbook postscript font
\usepackage{fontawesome}

% Enable link
\usepackage{hyperref}

% Enable condition ifthenelse
\usepackage{ifthen}

% Enable spacing
\usepackage{scrextend}

% ------------------------------------------------------------------------------
% Customizations - pt.2

% - Update section header
\renewcommand{\section}[1]{%
  \vspace{0.3in}%
  \centerline{\uppercase{\bf{#1}}}%
  \vspace{-6pt}}

% - Award line
\newcommand{\lineaward}[2]{%
  #1 \faTrophy{} #2\\[1ex]}

% - Presentation lines
\newcommand{\linepresentation}[3]{%
  #3 \hfill #1\\[.75ex]}

% - Employment header lines
\newcommand{\linehead}[2]{%
  {\bf #1} \hfill #2\\}
\newcommand{\linetitle}[1]{%
  {\sl #1}}

% - Volunteer line
\newcommand{\linevolunteer}[3]{%
  {\sl #3}, {\bf #2} \hfill #1}%\\[.5ex]}

% - define new environment for publications
\newenvironment{publicationlists}{%
  \begin{itemize}\leftmargin=3em \itemindent=-1em \itemsep=2pt%
  }{%
  \end{itemize}}

% - Define achievement & awards section
\newcommand{\awarditem}[3]{%
  \ifthenelse{\equal{#2}{}}
    {{\bf #1}}
    {{\bf #1}\hfill#2}
    \\ \vspace{-1em}
    \begin{addmargin}[2em]{3em}% 2em left, 3em right
    #3
    \end{addmargin}
    \vspace{-1em}
    }

% - update itemize have smaller itemsep
\let\olditemize\itemize
\renewcommand{\itemize}{%
  \olditemize\addtolength{\itemsep}{-4pt}}

% - url to roman (serif) font
\urlstyle{rm}


% ==============================================================================
\begin{document}
\begin{resume}

% \thispagestyle{empty} % no header on first page

% ------------------------------------------------------------------------------
\vspace{-1em}
\begin{center}
  {\it As a MSc in Computational Biology, I do Data Science work applied to biological data. I am passionate about Data Science, Machine Learning, Text Mining, and Data Visualization. Fluent in Python and R.}
\end{center}
\vspace{-1.5em}

\section{Projects in Data science}

\linehead{100 Days of Code}{Present}
\linetitle{See my technical blog: \url{http://csiu.github.io/blog/tag/100daysofcode}}
\begin{itemize}
  \item Used machine learning methods from Scikit-learn to build models and enter Kaggle competitions
  \item Experimented with Python (eg. Pandas, Matplotlib, BeautifulSoup, Nltk), D3, APIs, PostgreSQL
\end{itemize}

\linehead{MSc thesis: Characterization of the human thyroid epigenome}{Sept 2015 -- Present}
\linetitle{Michael Smith Genome Sciences Centre, Jones Lab}
\begin{itemize}
  \item Built a data pipeline using new \& existing code to automatically run \& replicate experimental analyses
  eg. modelling of chromatin states from epigenetic features using hidden Markov models
  \item Developed a novel metric for model selection \& implemented it on Github as an R package (hmmpickr)
\end{itemize}

\linehead{Literature-based knowledge discovery of biomedical text}{May 2015 -- Aug 2015}
\linetitle{Michael Smith Genome Sciences Centre, Jones Lab}
\begin{itemize}
  \item Developed text mining/machine learning approaches to prioritize therapeutics from unstructured text
  \item Created a visulization platform for interfacing with results stored in remote databases
\end{itemize}

\section{Work Experience}

{\bf Computational Biology Research Assistant} \hfill Sept 2013 -- Aug 2014\\
{\it BC Cancer Agency, Shah Lab}
\begin{itemize}
  \item Preprocessed data, developed pipelines, and wrapped third party software using R, Python and Bash
  \item Produced data visualizations and limited text suitable for scientific manuscripts with R (eg. ggplot2)
  \item Initiated pipelines on compute clusters to generate data for downstream analysis in Python and Bash
  \item Following an internship, promoted to full-time
\end{itemize}

{\bf Undergraduate Research Student} \hfill May 2013 -- Aug 2013\\
{\it UBC Department of Statistics}
\begin{itemize}
  \item Worked with collaborators to develop an R package using version control (Git, Github)
  \item Practiced good software development and continuous integration by writing unit tests for the R package
\end{itemize}

\section{Volunteer experience}
\linevolunteer{Oct 2016 \& Feb 2017}{%
  Software Carpentry workshop}{%
  Volunteer}
\begin{itemize}
  \item Troubleshot and helped students learn Git, GitHub, Bash, Python, and Jupyter Notebook
\end{itemize}

\section{Education}
\linehead{Master of Science, Computational Biology}{Sept 2014 -- Present}
University of British Columbia, Vancouver, BC
\begin{itemize}
  \item 10\% acceptance rate and awarded \$50,000+ in scholarships and travel grants
  \item Took graduate-level courses in Machine Learning, Data Mining, and Design and Analysis of Algorithms
\end{itemize}
\pagebreak

\end{resume}
\end{document}
