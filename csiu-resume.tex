% LaTeX file for resume
% This file uses the resume document class (res.cls)
\documentclass{res}

\name{\huge{Celia Siu}\\[4pt]}
\address{%
  % \centerline{%
  %   \faMobile{ CELLPHONE }%
  %   \faEnvelopeO{ EMAIL }}\\%
  % \centerline{%
  %   \faLinkedinSquare{ https://ca.linkedin.com/in/celiasiu }}\\%
  % \centerline{%
  %   \faGithubSquare{ http://github.com/csiu }%
  %   \faStackOverflow{ http://stackoverflow.com/users/2364901/csiu }}\\%
  \rule{\textwidth}{0.4pt}{}}


% Customize spacing
\newsectionwidth{0pt}  % So the text is not indented under section headings
\topmargin=-0.5in      % start text higher on the page

% Customize the header
\usepackage{fancyhdr}
\pagestyle{fancy}                  % set pagestyle for document
\renewcommand{\headrulewidth}{0pt} % suppress line drawn by default by fancyhdr
% - Adjust spacing
\setlength{\headheight}{24pt} % allow room for 2-line header
\setlength{\headsep}{24pt}    % space between header and text
% - Define headers and footers
\rhead{ {\it csiu}\\{\it p. \thepage} }
\cfoot{}

% Enable use of packages
%\usepackage{helvetica} % uses helvetica postscript font (download helvetica.sty)
%\usepackage{newcent}   % uses new century schoolbook postscript font
\usepackage{fontawesome}


% ------------------------------------------------------------------------------
% Customizations - pt.2

% - Update section header
\renewcommand{\section}[1]{%
  \vspace{0.3in}%
  \centerline{\uppercase{\bf{#1}}}%
  \vspace{-6pt}}

% - Award line
\newcommand{\lineaward}[2]{%
  #1 \faTrophy{} #2\\[1ex]}

% - Presentation lines
\newcommand{\linepresentation}[3]{%
  #3 \hfill #1\\[.75ex]}

% - Employment header lines
\newcommand{\linehead}[2]{%
  {\bf #1} \hfill #2\\}
\newcommand{\linetitle}[1]{%
  {\sl #1}}

% - Volunteer line
\newcommand{\linevolunteer}[3]{%
  {\sl #3}, {\bf #2} \hfill #1\\[.5ex]}

% - define new environment for publications
\newenvironment{publicationlists}{%
  \begin{itemize}\leftmargin=3em \itemindent=-1em \itemsep=2pt%
  }{%
  \end{itemize}}

% - update itemize have smaller itemsep
\let\olditemize\itemize
\renewcommand{\itemize}{%
  \olditemize\addtolength{\itemsep}{-4pt}}


% ==============================================================================
\begin{document}
\begin{resume}

\thispagestyle{empty} % no header on first page

% ------------------------------------------------------------------------------
\section{Education}
\linehead{Pursuing a Master's degree, Bioinformatics}{Sept.2014 - Present}\vspace{-8pt}
\begin{itemize}
  \item University of British Columbia, Vancouver, BC
  \item Supervised by Dr. Steven Jones at the Genome Sciences Centre, BC Cancer Agency
  \item 4 month lab rotations with Steven Jones, Wyeth Wasserman, and Denise Daley
\end{itemize}

\linehead{Bachelor of Science, Biochemistry}{Sept.2008 -- May.2013}\vspace{-8pt}
\begin{itemize}
  \item University of British Columbia, Vancouver, BC
  \item Completed the Co-operative Education Program
\end{itemize}

% ------------------------------------------------------------------------------
\section{M.Sc. Thesis Project}
``Characterization of the normal thyroid reference epigenome"\\
In this study, we compare the consistency of chromatin state annotations across the epigenomes from the grossly uninvolved tumour-adjacent thyroid tissue of four human individuals using ChIP-sequencing and RNA-sequencing data. We examine 6 histone modifications, identify chromatin states using hidden Markov models, produce a new metric for model selection, and establish epigenomic maps of 19 chromatin states.

% ------------------------------------------------------------------------------
\section{Awards}

\begin{center}
  \begin{tabular}{l}
    \lineaward{2016}{CEEHRC Travel Award}
    \lineaward{2016}{John Bosdet Memorial Fund}
    \lineaward{2015}{NSERC Canada Graduate Scholarship Master's Award}
    \lineaward{2014}{CIHR Strategic Training in Health Research in Bioinformatics Award}
    \lineaward{2013}{NSERC Undergraduate Student Research Award (USRA)}
    \lineaward{2008}{UBC President's Entrance Scholarship}
  \end{tabular}
\end{center}
\vspace{-1em}

% ------------------------------------------------------------------------------
\section{Publications}

Co-authored in:\vspace{4pt}
\begin{publicationlists}
  \item McPherson et al. (2016) Divergent modes of clonal spread and intraperitoneal mixing in high-grade serous ovarian cancer. {\sl Nature Genetics}. doi:10.1038/ng.3573
  \item Eirew et al. (2015) Dynamics of genomic clones in breast cancer patient xenografts at single-cell resolution. {\sl Nature, 518}(7539):422-6. doi: 10.1038/nature13952
  \item Leon et al. (2012) Application and evaluation of automated methods to extract neuroanatomical connectivity statements from free text. {\sl Bioinformatics, 28}(22):2963-70. doi: 10.1093/bioinformatics/bts542
\end{publicationlists}

Acknowledged in:\vspace{4pt}
\begin{publicationlists}
  \item Zoubarev et al. (2012) Gemma: a resource for the reuse, sharing and meta-analysis of expression profiling data. {\sl Bioinformatics, 28}(17):2272-3. doi: 10.1093/bioinformatics/bts430
\end{publicationlists}

% ------------------------------------------------------------------------------
\section{Presentations}

\linepresentation{July 9 \& 10, 2016}{%
  Bioinformatic characterization of the normal thyroid reference epigenome}{%
  Poster presentation, Intelligent Systems for Molecular Biology (ISMB) 2016 Conference}
\linepresentation{March 11, 2016}{%
  Characterization of the normal reference thyroid epigenome}{%
  Poster presentation, BTP/IOP/GSAT 2016 Retreat}
% \linepresentation{July 24, 2015}{%
%   Literature-based knowledge discovery of Biomedical Text}{%
%   Oral presentation, Bioinformatics Research Rotation Presentation}
% \linepresentation{March 27, 2015}{%
%   miRNA promoter recognition with CAGE and sRNA-seq}{%
%   Oral presentation, Bioinformatics Research Rotation Presentation}
% \linepresentation{November 28, 2014}{%
%   Primer Data Extraction and Blast API}{%
%   Oral presentation, Bioinformatics Research Rotation Presentation}
\vspace{-1.5em}

% ------------------------------------------------------------------------------
\section{Employment history}

\linehead{BCCRC $|$ The University of British Columbia}{Jan.2014 -- Aug.2014}
\linetitle{Research Assistant at the Shah Lab}
\begin{itemize}
  \item Wrote Bash, Python, and R scripts for downstream analysis and pipeline development
  \item Produced figures and limited text suitable for scientific manuscripts
  \item Leveraged parallel computing for analyses on a compute cluster to increase efficiency and decrease runtime
\end{itemize}

\linehead{BCCRC $|$ Provincial Health Services Authority}{Sept.2013 -- Dec.2013}
\linetitle{Research Intern at the Shah Lab}
\begin{itemize}
  \item Ran automated analysis pipelines for data processing and data generation
  \item Wrote Bash and Python scripts to rename files, parse data, and wrap third party software
  \item Documented observations and comments on platforms such as the Atlassian Confluence Wiki, Atlassian Jira issue tracker, and Atlassian Stash Git management system
\end{itemize}

\linehead{NSERC Undergraduate Student Research Award (USRA)}{May.2013 -- Aug.2013}
\linetitle{Undergraduate Summer Research Student in the UBC Department of Statistics}
\begin{itemize}
  \item Contributed to the development of an R toolkit for the management and delivery of data analytic course and research material
  \item Organized, within a budget, a small meet-and-greet coffee social to acquaint oneself with the UBC Statistics NSERC-USRA students and their supervisors
\end{itemize}

\linehead{UBC Centre for High-throughput Biology}{May.2012 -- Aug.2012}
\linetitle{Linux System Administrator Assistant}
\begin{itemize}
  \item Created and updated documentation and protocols for internal use and for end users
  \item Created and setup user accounts on desktops, servers, and wikis
  \item Used a variety of softwares -- including Nagios, Bacula, and CFengine -- to monitor, backup, and perform system administrative jobs
  \item Provided troubleshooting services for users and kept record of the issues, challenges, solutions in Request Tracker
\end{itemize}

\linehead{UBC Centre for High-throughput Biology}{Sept.2011 -- Apr.2012}
\linetitle{Bioinformatics Research Assistant at the Pavlidis Lab}
\begin{itemize}
  \item Independently processed and checked the quality of microarray data for the FishManOmics project headed by the Department of Fisheries and Oceans Canada
  \item Paid attention to detail to analyze, annotate, and curate Gene Expression Omnibus (GEO) datasets that were to be imported into the Gemma database
  \item Reported issues based on continuous evaluation of lab software (e.g. Gemma) across multiple operating systems including Linux, Windows, and Mac OS
\end{itemize}

% \linehead{Biofine International Inc.}{May.2011 -- Aug.2011}
% \linetitle{Organic Chemistry Synthesis Laboratory Assistant}
% \begin{itemize}
%   \item Verbally communicated with supervisors to receive instructions for the execution of experiments
%   \item Researched, experimented, and evaluated the method of Marfey's reaction for enantiomeric resolution of amino acids
%   \item Multitasked in coordination between executing experiments, keeping records, restocking supplies, and cleaning glassware
% \end{itemize}

% \linehead{A\&W Food Services of Canada}{July.2007 -- Oct.2010}
% \linetitle{Cashier}

% \linehead{Pacific National Exhibition}{Aug.2006, Aug.2007, Aug.2008, Aug.2009}
% \linetitle{Concessions Attendant}

% -----------------------------------------------------------------------------
\section{Volunteer experience}

\linevolunteer{Oct.2016}{%
  Software Carpentry workshop}{%
  Helper}
\linevolunteer{Feb.2013 -- Apr.2013}{%
  St. Paul's Hospital}{%
  Wayfinder}
\linevolunteer{May.2013 -- Aug.2013}{%
  St. Paul's Hospital}{%
  Healthy Heart Resource Centre Clerk}

% -----------------------------------------------------------------------------
% \section{Computing skills}
% \begin{center}
%   Experienced in R, Python, Bash shell scripting,
%   Emacs, Git/Github, HTML, LaTeX,
%   Unix/Linux, Mac, Windows,
%   Microsoft Word/Excel/PowerPoint, Trouble-shooting
% \end{center}

\end{resume}
\end{document}
